%%%%%%%%%%%%%%%%%%%%%%%%%%%%%%%%%%%%%%%%%
% Dictionary
% LaTeX Template
% Version 1.0 (20/12/14)
%
% This template has been downloaded from:
% http://www.LaTeXTemplates.com
%
% Original author:
% Vel (vel@latextemplates.com) inspired by a template by Marc Lavaud
%
% License:
% CC BY-NC-SA 3.0 (http://creativecommons.org/licenses/by-nc-sa/3.0/)
%
%%%%%%%%%%%%%%%%%%%%%%%%%%%%%%%%%%%%%%%%%

%----------------------------------------------------------------------------------------
%	PACKAGES AND OTHER DOCUMENT CONFIGURATIONS
%----------------------------------------------------------------------------------------

\documentclass[10pt,a4paper]{ctexart} % 10pt font size, A4 paper and two-sided margins

\usepackage[top=3.5cm,bottom=3.5cm,left=3.7cm,right=4.7cm,columnsep=30pt]{geometry} % Document margins and spacings

\usepackage[utf8]{inputenc} % Required for inputting international characters
\usepackage[T1]{fontenc} % Output font encoding for international characters

\usepackage{palatino} % Use the Palatino font

\usepackage{microtype} % Improves spacing

\usepackage{multicol} % Required for splitting text into multiple columns

\usepackage[bf,sf,center]{titlesec} % Required for modifying section titles - bold, sans-serif, centered

\setCJKfamilyfont{lxgw}{LXGWWenKai-Regular.ttf}

\usepackage{fancyhdr} % Required for modifying headers and footers
\fancyhead[L]{\textsf{\rightmark}} % Top left header
\fancyhead[R]{\textsf{\leftmark}} % Top right header
\renewcommand{\headrulewidth}{1.4pt} % Rule under the header
\fancyfoot[C]{\textbf{\textsf{\thepage}}} % Bottom center footer
\renewcommand{\footrulewidth}{1.4pt} % Rule under the footer
\pagestyle{fancy} % Use the custom headers and footers throughout the document

\newcommand{\entry}[2]{\markboth{#1}{#1}\textbf{#1}:\ \CJKfamily{lxgw}{#2}}  % Defines the command to print each word on the page, \markboth{}{} prints the first word on the page in the top left header and the last word in the top right

%----------------------------------------------------------------------------------------

\begin{document}

\section*{\CJKfamily{lxgw}{ACM/ICPC 大赛常见英语词汇}}

\section*{A}

\begin{multicols}{2}

\entry{abbreviation}{[数学] 约分}

\entry{activity on edge}{AOE网}

\entry{activity on vertex}{AOV网}

\entry{add, subtract, multiply and divide}{加减乘除}

\entry{adjacency list}{邻接表(adjacency multilist 邻接多重表)}

\entry{adjacency matrix}{邻接矩阵}

\entry{adjacent sequence elements}{相邻的元素串}

\entry{adjacent vertex}{相邻顶点}

\entry{algebraic term}{代数项}

\entry{alphabetical order}{字典序}

\entry{alternately rise and fall}{交替上升和下降}

\entry{Ambiguous}{模糊不清}

\entry{ancestor}{祖先}

\entry{anticlockwise}{逆时针}

\entry{Approximate String Matching}{模糊匹配}

\entry{Arbitrary Precision Arithmetic}{高精度计算}

\entry{arc}{弧}

\entry{arithmetic mean}{算数平均值}

\entry{arithmetic progression}{等差数列(geometric progression 等比数列)}

\entry{array}{数组}

\entry{articulation point}{连接点}

\entry{ascending lexicographical order}{词典顺序升序排列}

\entry{ascending order}{升序(descending order 降序)}

\entry{aspect ratio}{固定长宽比}

\entry{assemble}{组合}

\entry{assess}{评定,评估}

\entry{assigned}{指定的,赋值的}

\entry{augmenting path graph}{增广路径图(augmenting path 增广路径)}

\entry{average search length}{平均查找长度}

\entry{average temperature}{顺时针}

\entry{axis}{axes 轴}

\end{multicols}

\section*{B}

\begin{multicols}{2}

\entry{balance merging sort}{平衡归并排序}

\entry{balance two-way merging}{二路平衡归并排序}

\entry{Bandwidth Reduction}{带宽压缩}

\entry{balanced binary tree}{平衡二叉树}

\entry{base}{底边;幂的底数}

\entry{biconnected graph}{重连通图}

\entry{bidirectional}{双向的}

\entry{binary search tree}{二叉查找树}

\entry{binary search}{二分查找}

\entry{binary sort tree}{二叉排序树}

\entry{binary}{二进制}

\entry{bipartite graph}{二部图}

\entry{Bishop}{主教(象)只斜走。格数不限不能越子。每方有两象,黑白格各占1个}

\entry{blank string}{空白(空格)串}

\entry{block search}{分块查找}

\entry{boundary}{界限}

\end{multicols}

\section*{C}

\begin{multicols}{2}

\entry{calculate}{计算}

\entry{Calendrical Calculations}{日期}

\entry{carpet}{地毯}

\entry{chariot}{战车(中国象棋)}

\entry{checkmate}{(国际象棋) 将死, 输棋,将死, 败局, 败北,挫败,}

\entry{circular linked list}{循环链表}

\entry{circular queue}{循环队列}

\entry{Clique}{最大团}

\entry{clockwise order}{顺时针方向顺序(anticlockwise 逆时针)}

\entry{Coefficient}{系数,率,程度}

\entry{Collinear}{共线的}

\entry{column major order}{以列为主的顺序分配}

\entry{columns}{列}

\entry{Combinatorial Problems}{组合问题}

\entry{comma}{逗号}

\entry{common superstring}{公共父串}

\entry{compile}{v 编译,汇编}

\entry{complete binary tree}{完全二叉树}

\entry{complete graph}{完全图}

\entry{composite numbers}{合数}

\entry{Computational Geometry}{计算几何}

\entry{concave}{凹的}

\entry{connected component}{连通分量(Connected Components 连通分支)}

\entry{consecutive}{连续的}

\entry{constant}{常数,常量 不变的,始终如一的,持续不断的}

\entry{Constrained and Unconstrained Optimization}{最值问题}

\entry{Convex Hull}{凸包}

\entry{coordinates}{坐标}

\entry{corrupt}{腐烂,破坏}

\entry{counterclockwise}{逆时针}

\entry{critical path}{关键路径}

\entry{Cryptography}{密码}

\entry{Cube root}{立方根}


\end{multicols}

\section*{D}

\begin{multicols}{2}

\entry{D is rounded to 2 decimal places}{D是精确到小数点后2位}

\entry{Data Structures}{基本数据结构}

\entry{data type}{数据类型}

\entry{decimal}{小数 小数的,十进制的}

\entry{decision tree}{判定树}

\entry{Deck}{甲板}

\entry{define}{v 定义,明确,使规定}

\entry{deformed}{变形的}

\entry{Denominator}{分母}

\entry{denote}{代表, 指代, 预示, 意思是,标志;象征}

\entry{dense graph}{稠密图}

\entry{Deployed}{部署}

\entry{depth}{深度}

\entry{deque}{(double-ended queue) 双端列表}

\entry{descendant}{子孙}

\entry{destination}{终点}

\entry{Determinants and Permanents}{行列式}

\entry{diagonal}{对角(diagonally 斜对角线的)}

\entry{dial}{钟面,拨打}

\entry{dialing}{拨号音 打电话,拨电话号码( dial的现在分词 )}

\entry{Dictionaries}{字典}

\entry{difference}{差}

\entry{digital analysis method}{数字分析法}

\entry{digital search tree}{数字查找树}

\entry{digit}{位数,数字}

\entry{digraph}{(directed graph) 有向图}

\entry{Dimensional}{尺寸}

\entry{diminishing increment sort}{随小增量排序}

\entry{direct access file}{直接存取文件}

\entry{directed acyclic graph}{有向无环图}

\entry{directory structure}{目录结构}

\entry{directory}{(计算机文件或程序的)目录,指导的咨询的, 管理的}

\entry{discrete Fourier transform}{离散傅里叶变换}

\entry{disjoint}{不相交的}

\entry{Distinct values}{独一无二的值}

\entry{distinct}{不同的;独一无二的}

\entry{division method}{除法}

\entry{divisor}{因子,分母}

\entry{doubly linked list}{双向链表}

\entry{doubly linked tree}{双链树}

\entry{Drawing Graphs Nicely}{图的描绘}

\entry{Drawing Trees}{树的描绘}

\entry{duplicated}{复制;打印的}

\entry{duplicates}{完全一样的东西,复制品( duplicate的名词复数 )}
\end{multicols}

\section*{E}

\begin{multicols}{2}

\entry{Edge and Vertex Connectivity}{割边/割点}

\entry{Edge Coloring}{边染色}

\entry{embed}{插入}

\entry{enable}{启用}

\entry{Entry}{进口}

\entry{equation}{方程式,等式}

\entry{equivalent equation}{同解方程,等价方程}

\entry{equivalent}{相等的,等效的}

\entry{estimate}{预测}

\entry{Eulerian Cycle / Chinese Postman}{Euler回路/中国邮路}

\entry{evaluate}{v 评价,估价}

\entry{evaluated}{求···的值}

\entry{even}{偶数的}

\entry{excluding}{排除,拒绝( exclude的现在分词), 驱逐,除…外,不包括}

\entry{execute}{v 执行,完成}

\entry{executed}{执行的;生效的}

\entry{exponent}{指数;幂}

\entry{external sort}{外部排序}

\end{multicols}

\section*{F}

\begin{multicols}{2}

\entry{Facility}{设备,设施}

\entry{factorial}{阶乘, 因子的,阶乘的}

\entry{Factoring and Primality Testing}{因子分解/质数判定}

\entry{Feedback Edge/Vertex Set}{最大无环子图}

\entry{Finite State Machine Minimization}{有穷自动机简化}

\entry{fixed-aggregate data type}{固定聚合数据类型}

\entry{foggiest idea}{概念}

\entry{folding method}{折叠法}

\entry{follow by}{跟随,其后}

\entry{forest}{森林}

\entry{formula}{公式}

\entry{fraction}{分数,小部分}

\entry{front}{队头}

\entry{full binary tree}{满二叉树}


\end{multicols}

\section*{G}

\begin{multicols}{2}

\entry{gcd}{(greatest common divisor) 最大公约数}

\entry{generalized list}{广义表}

\entry{Generating Graphs}{图的生成}

\entry{Generating Partitions}{划分生成}

\entry{Generating Permutations}{排列生成}

\entry{Generating Subsets}{子集生成}

\entry{geometric progression}{等比数列}

\entry{grabh}{图}

\entry{Graph Data Structures}{图}

\entry{Graph Isomorphism}{同构}

\entry{Graph Partition}{图的划分}

\entry{Graph Problems — hard}{图论-NP问题}

\entry{Graph Problems — polynomial}{图论-多项式算法}

\entry{greatest integer}{最大整数}

\entry{grid}{网格;方格;(地图上的)坐标方格}


\end{multicols}

\section*{H}

\begin{multicols}{2}

\entry{Hamiltonian Cycle}{Hamilton回路}

\entry{hash search}{散列查找(hash table 散列表)}

\entry{head node}{头结点(head pointer 头指针)}

\entry{heap sort}{堆排序}

\entry{horizontal or vertical direction}{水平和垂直方向}

\entry{horizontally}{水平地}

\entry{horizontal}{水平的}

\entry{Huffman tree}{哈夫曼树}

\end{multicols}

\section*{I}

\begin{multicols}{2}

\entry{Identifier}{标识符,识别码}

\entry{immediate predecessor}{直接前趋(immediate successor 直接后继)}

\entry{immediately allocating method}{直接定址法}

\entry{improper fraction}{假分数}

\entry{in the range of}{在…范围内}

\entry{in the shape of a cross}{十字形}

\entry{incident edge}{关联边}

\entry{indegree}{入度}

\entry{indent}{缩进}

\entry{identical}{相同的}

\entry{Independent Set}{独立集}

\entry{indexed file}{索引文件}

\entry{indexed non-sequential file}{索引非顺序文件(indexed sequential file 顺序)}

\entry{indicating}{指示的,标志的}

\entry{inequality}{不等式}

\entry{infinite}{无限的}

\entry{initial}{最初的,词首的,开始的,首字母}

\entry{initial node}{初始结点}

\entry{initialization}{初始化,赋初值}

\entry{inorder traversal}{中序遍历}

\entry{insertion sort}{插入排序}

\entry{insertion}{插入}

\entry{integer}{整数}

\entry{Interior}{内部,本质}

\entry{internal sort}{内部排序}

\entry{Interpret}{解释,执行}

\entry{intersect}{v 相交,交叉}

\entry{intersection}{横断,横切, 交叉,相交, 交叉点,交叉线, [数] 交集,}

\entry{Intersection Detection}{碰撞测试}

\entry{intersection}{横断,横切,交叉}

\entry{intersect}{相交}

\entry{intervals}{间隔时间, 间隔( interval的名词复数 ), 区间}

\entry{Invade}{侵略}

\entry{invalid}{无效的}

\entry{inverted file}{倒排文件}

\entry{irreparably}{不能恢复地}

\end{multicols}

\section*{J}

\begin{multicols}{2}

\entry{Job Scheduling}{工程安排}

\entry{justified}{合理的,合法化的}

\end{multicols}

\section*{K}

\begin{multicols}{2}

\entry{Kd-Trees}{线段树}

\entry{Knapsack Problem}{背包问题}

\entry{Knight}{骑士(马)每步棋先横走或直走一格,然后再往外斜走一格;或者先斜走一格,最后再往外横走或竖走一格(即走“日”字)。可以越子,没有象棋中的“蹩马腿”限制。}

\end{multicols}

\section*{L}

\begin{multicols}{2}

\entry{lcm}{(Least Common Multiple) 最小公倍数}

\entry{left or right-justified}{左对齐 or 右对齐}

\entry{lexicographically}{字典序}

\entry{like terms ,similar terms}{同类项}

\entry{linear algebra}{线性代数 (linear equation 线性方程 linear linked list 线性链表)}

\entry{Linear Programming}{线性规划}

\entry{linear structure}{线性结构}

\entry{link field}{链域}

\entry{linked list}{链表}

\entry{literal coefficient}{字母系数}

\entry{logarithm}{对数}

\entry{logical structure}{逻辑结构}

\entry{Longest Common Substring}{最长公共子串}

\entry{loop}{环}

\end{multicols}

\section*{M}

\begin{multicols}{2}

\entry{Maintaining Line Arrangements}{平面分割}

\entry{master file}{主文件}

\entry{Matching}{匹配}

\entry{Matrix Multiplication}{矩阵乘法}

\entry{maximum matching}{最大匹配}

\entry{meadow}{草坪}

\entry{mean}{平均值}

\entry{Medial-Axis Transformation}{中轴变换}

\entry{Median and Selection}{中位数}

\entry{memorable}{值得纪念的, 显著的,难忘的, 重大的,著名的}

\entry{merge sort}{归并排序}

\entry{mid-square method}{平方取中法}

\entry{minimal}{最小限度的}

\entry{minimal volume}{最小体积}

\entry{minimum}{(cost) 最小(代价)生成树}

\entry{mixed number}{带分数}

\entry{mod}{v 求余}

\entry{modulus}{系数,模数}

\entry{Motion Planning}{运动规划}

\entry{motion}{多边形}

\entry{multi-dimensional array}{多维数组}

\entry{multilinked list}{多重链表}

\entry{multilist file}{多重链表文件}

\entry{multiple}{多重的,多样的,许多的,倍数}

\entry{multiplication}{乘法}

\entry{municipal}{市政的}

\end{multicols}

\section*{N}

\begin{multicols}{2}

\entry{Nearest Neighbor Search}{最近点对查询}

\entry{negative , positive}{负,正}

\entry{Network Flow}{网络流}

\entry{no special punctuation symbols or spacing rules}{无特殊标点符号或间距的规则}

\entry{non-intersecting}{非相交的, 不相交的,}

\entry{nonlinear structure}{非线性结构}

\entry{notation}{标记}

\entry{numerator}{分子}

\entry{numerical coefficient}{数字系数}

\entry{Numerical Problems}{数值问题}

\end{multicols}

\section*{O}

\begin{multicols}{2}

\entry{Obesity}{肥胖}

\entry{octal}{八进制的}

\entry{binhex}{十六进制}

\entry{odd and even}{奇和偶}

\entry{optimal}{最佳的}

\entry{optimally}{最佳}

\entry{Orbit}{轨道}

\entry{ordered pair}{有序对 (ordered tree 有序树)}

\entry{Ordinal}{有次序的}

\entry{original equation}{原方程}

\entry{origin}{原点}

\entry{orthogonal list}{十字链表}

\entry{Out degree}{出度}

\entry{Over brim}{溢出}

\entry{overflow}{上溢}

\entry{Overlapping}{覆盖}

\entry{ox}{牛}

\end{multicols}

\section*{P}

\begin{multicols}{2}

\entry{palindrome}{回文}

\entry{palindromic}{回文的}

\entry{parallel}{平行的}

\entry{parity property}{奇偶性}

\entry{partial order}{偏序}

\entry{Pawn}{禁卫军(兵)只能向前直走,每次只能走一格。但走第一步时,可以走一格或两格。兵的吃子方法与行棋方向不一样,它是直走斜吃,即如果兵的斜进一格内有对方棋子,就可以吃掉它而占据该格}

\entry{physical structure}{物理结构}

\entry{Pipe}{管道}

\entry{Planarity Detection and Embedding}{平面性检测和嵌入}

\entry{polygon-shaped faces}{多边形}

\entry{polyphase merging sort}{多步归并排序}

\entry{Point Location}{位置查询}

\entry{pointer field}{指针域}

\entry{Polygon Partitioning}{多边形分割}

\entry{positive and negative integers}{正整数和负整数}

\entry{postorder traversal}{后序遍历}

\entry{precision}{精密,精确度}

\entry{predecessor}{前趋}

\entry{prefix}{前缀}

\entry{preorder traversal}{先序遍历}

\entry{prime}{质数}

\entry{Priority Queues}{优先队列}

\entry{proceed}{运行}

\entry{process}{v 加工,处理,程序,进程}

\entry{process a sequence of,distinct integers}{处理一串,个不同的整数}

\entry{profile}{轮廓}

\entry{proper fraction}{真分数}

\entry{proportional}{成比例的}

\entry{Protrusions}{凸起物}

\entry{Pyramid}{金字塔,渐增}

\end{multicols}

\section*{Q}

\begin{multicols}{2}

\entry{quadrant}{象限,四分之一圆}

\entry{Queen}{皇后 横、直、斜都可以走,步数不受限制,但不能越子}

\entry{quotient}{商}

\end{multicols}

\section*{R}

\begin{multicols}{2}

\entry{radix sort}{基数排序}

\entry{Random Number Generation}{随机数生成}

\entry{random number method}{随机数法}

\entry{Range Search}{范围查询}

\entry{rat, ox, tiger, rabbit, dragon, snake, horse, sheep, monkey, rooster, dog, pig}{十二生肖}

\entry{rate of convergence}{收敛速度}

\entry{rear}{队尾}

\entry{rectangular}{矩形的,成直角的}

\entry{Relates}{叙述,讲述}

\entry{replacement selection sort}{置换选择排序}

\entry{respectively}{各自的,分别的,独自的}

\entry{robustness}{鲁棒性}

\entry{Rook}{战车 横竖均可以走,步数不受限,不能斜走。除王车易位外不能越子。}

\entry{rooster}{鸡}

\entry{root sign}{根号}

\entry{round()}{四舍五入(当取舍位为5时,若取舍位数前的小数为奇数则直接舍弃,若为偶数则向上取舍)}

\entry{rounded to,decimal places}{精确到小数点后,位}

\entry{row major order}{以行为主的顺序分配}

\entry{Rows and columns}{行与列}

\end{multicols}

\section*{S}

\begin{multicols}{2}

\entry{Satisfiability}{可满足性}

\entry{scenario}{方案;(可能发生的)情况;}

\entry{search}{(sequential search) 线性查找(顺序查找)}

\entry{searching}{查找,线索}

\entry{segment}{段,分割}

\entry{segment}{环节, 部分,分段, 分割,划分,}

\entry{selection sort}{选择排序}

\entry{semicolon}{分号}

\entry{sequence}{顺序,序列,连续}

\entry{serial}{连续的, 连载的, 顺序排列的,}

\entry{series}{连续的同类事物,系列}

\entry{series}{系列}

\entry{Set and String Problems}{集合与串的问题}

\entry{Set Cover}{集合覆盖}

\entry{Set Data Structures}{集合}

\entry{Set Packing}{集合配置}

\entry{Shape Similarity}{相似多边形}

\entry{Shell}{贝壳,脱壳}

\entry{shelter}{遮蔽物}

\entry{Shortest Common Superstring}{最短公共父串}

\entry{Shortest Path}{最短路径}

\entry{simple cycle}{简单回路(simple path 简单路径)}

\entry{Simplifying Polygons}{多边形化简}

\entry{simultaneously}{同时的}

\entry{single linked list}{单链表}

\entry{sink}{汇点}

\entry{solution}{解决方案}

\entry{Solving Linear Equations}{线性方程组}

\entry{source}{源点}

\entry{spanning forest}{生成森林}

\entry{spanning tree}{生成树}

\entry{spares graph}{稀疏图}

\entry{sparse matrix}{稀疏矩阵}

\entry{specify}{指定}

\entry{square root}{平方根}

\entry{square}{平方,正方形,广场,方格}

\entry{Squared}{平方}

\entry{Stack Overflow}{堆栈溢出通常是您的程序陷入了无穷递归,或递归嵌套层数过多。}

\entry{statistical}{统计的}

\entry{Steiner Tree}{Steiner树}

\entry{stem}{词根}

\entry{String Matching}{模式匹配}

\entry{strongly connected graph}{强连通图}

\entry{subgraph}{子图}

\entry{subsequent}{随后的,后来的}

\entry{substring}{子串(subtree 子树)}

\entry{successor}{后继}

\entry{sufficient}{充足的;足够的;}

\entry{suffix}{后缀}

\entry{Supervisor}{监督人}

\entry{symmetric matrix}{对称矩阵}

\entry{tail pointer}{尾指针}

\entry{terminal node}{终端结点}

\entry{Text Compression}{压缩}

\entry{threaded binary tree}{线索二叉树}

\end{multicols}

\section*{T}

\begin{multicols}{2}

\entry{times}{乘}

\entry{Topological Sorting}{拓扑排序}

\entry{toss}{扔(硬币)}

\entry{Transitive Closure and Reduction}{传递闭包}

\entry{transposed matrix}{转置矩阵}

\entry{traversal of tree}{树的遍历}

\entry{traversing binary tree}{遍历二叉树}

\entry{traversing graph}{遍历图}

\entry{tree index}{树型索引}

\entry{triangle}{三角形}

\entry{triangle inequality}{三角不等式}

\entry{Triangulation}{三角剖分}

\entry{triple}{三倍的,三方的,三部分的, 增至三倍, 三倍的数[量], 三个一组,}

\entry{Tromino}{三格骨牌}

\entry{Troop}{军队,组群}

\entry{triangular matrix}{三角矩阵}

\entry{two adjacent sequence elements}{两个相邻的元素串}

\entry{two-dimensional array}{二维数组}

\entry{two-dimensional}{维数}

\end{multicols}

\section*{U}

\begin{multicols}{2}

\entry{ultimate}{基本的,终极的}

\entry{unconnected graph}{非连通图}

\entry{underflow}{下溢}

\entry{undigraph}{(undirected graph) 无向图}

\entry{union}{并集}

\entry{unique identifier}{唯一的标识符}

\entry{unordered pair}{无序对(unordered tree 无序树)}

\entry{uppercase}{大写字母盘, 以大写字母印刷, 大写字母的}

\entry{uppercase}{(Capital) 大写字母(Lowercase letters 小写字母)}

\end{multicols}

\section*{V}

\begin{multicols}{2}

\entry{variable-aggregate data type}{可变聚合数据类型}

\entry{variable}{变量}

\entry{Vertex Coloring}{点染色(Vertex Cover 点覆盖)}

\entry{vertex}{顶点,最高点}

\entry{vertical}{垂直线,垂直面 垂直的,顶点的}

\entry{volume}{数量,容量}

\entry{Voronoi Diagrams}{Voronoi图}

\entry{vulnerable}{容易受到攻击的}

\end{multicols}

\section*{W}

\begin{multicols}{2}

\entry{weakly connected graph}{弱连通图}

\entry{weight}{权}

\entry{weighted average}{加权平均值}

\entry{weighted graph}{加权图}

\entry{wooden planks}{木板}

\end{multicols}

\end{document}